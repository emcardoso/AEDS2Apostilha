\documentclass[a4paper,11pt]{book}
\usepackage[utf8]{inputenc}
\usepackage[brazil]{babel}
\usepackage{amsmath}
\usepackage{amsfonts}
\usepackage{listings}
\usepackage{graphicx}
\usepackage{wasysym}
\usepackage{amssymb}
\usepackage{color}
\usepackage[usenames,dvipsnames,svgnames,table]{xcolor}

\renewcommand{\lstlistingname}{Código}

\parindent0pt  \parskip10pt             % make block paragraphs

\title{\bf Algoritmos e Estruturas de Dados II}    % Supply information
\author{Elton M. Cardoso}              %   for the title page.
\date{\today}            


\definecolor{codegreen}{rgb}{0,0.6,0}
\definecolor{codegray}{rgb}{0.5,0.5,0.5}
\definecolor{codeblue}{rgb}{0.02,0.05,0.70}
\definecolor{codepurple}{rgb}{0.58,0,0.82}
\definecolor{backcolour}{rgb}{0.96,0.96,0.93}

\definecolor{numbercolour}{rgb}{0.0,0.0,0.0}
\definecolor{frmcolour}{rgb}{0.90,0.90,0.90}

\lstdefinestyle{mystyle}{
    backgroundcolor=\color{backcolour},
    framexleftmargin=4mm, 
    frame=shadowbox,
    %frameshape={RYRYNYYYY}{yny}{yny}{RYRYNYYYY},
    rulesepcolor=\color{frmcolour},
    commentstyle=\color{codegreen},
    keywordstyle=\color{codeblue},
    numberstyle=\tiny\color{numbercolour},
    stringstyle=\color{codepurple},
    basicstyle=\footnotesize,
    breakatwhitespace=false,         
    breaklines=true,                 
    captionpos=b,                    
    keepspaces=true,                 
    numbers=left,                    
    numbersep=5pt,                  
    showspaces=false,                
    showstringspaces=false,
    showtabs=false,                  
    tabsize=2
}
 
\lstset{style=mystyle}

\begin{document}

\frontmatter                            % only in book class (roman page #s)
\maketitle                              % Print title page.
\tableofcontents                        % Print table of contents
\mainmatter       

\chapter{Introdução}
   
   Em sua concepção, os primeiros computadores possuíam um desenho muito mais focado no cálculo (computação)
   do que no armazenamento de dados propriamente ditos. Foi em 1945 que surgiu uma arquitetura de computação que separava
   a unidade de processamento da memória, a arquitetura \emph{Von Neummann}. Nessa nova arquitetura dados são separados 
   do programa, podendo potencialmente serem armazenados em um módulo de memória fora da unidade de processamento.
   
   Computadores modernos são equipados com níveis de memória que podem ser:
   \begin{itemize}
     \item Registradores: São unidades de memória que armazenam resultados intermediário das operações realizadas pelo processador.
     Em geral são limitados a poucas unidades e são situados dentro do processador. 
     \item Memória \emph{cache} :  É um memória intermediária entre os registradores e a memória principal para melhor eficiência.
     A memória \emph{cache} é situada dentro do próprio processador. 
     \item Memória principal: A memória principal, diferentemente das duas primeiras, é situada fora do processador e é conectada a ele
     por meio de barramentos de dados.
   \end{itemize}

   Tantos registradores, memória \emph{cache} e memória principal são memórias voláteis e não capazes de reter os dados 
   após o sistema ser desligado. Há no entanto uma memória não volátil, também chamada de memória secundária, que  preserva seu 
   conteúdo após ser desligada. 
      
   Para acessar dados tanto na memória principal quanto na memória secundária a unidade de processamento necessita se comunicar
   com essas memórias, por meio de um barramento, informando quais dados deve ser transferidos para a memória \emph{cache} ou 
   para o os registradores. O processo de transferência dos dados em si pode ser uma operação mais complexa que pode requerer 
   uma certa quantidade de tempo para se concretizar.
   
   Idealmente processador e memórias deveriam operar de modo que o processador não tivesse que esperar até que um dado requisitado 
   fosse transferido para a memória interna do processador. Na prática isso não acontece e o processo de transferência dos dados
   pode levar vários ciclos de operação do processador.
   
   O problema é agravado quando o dispositivo de memória externa é lento e requere muito tempo para acessar e recuperar os dados, como
   acontece por exemplo com discos rígidos. Em geral memórias secundárias são mais lentas que a memória principal, a ponto
   de ser necessário levar esse tempo em consideração para evitar  que o  o que nos leva ao tópico
   central desse texto. Técnicas para recuperação eficiente de dados em memória secundária. 
   
   
   Antes que o assunto seja apresentada de modo mais aprofundado, vamos apresentar os conceitos preliminares necessários. 
   
   
   \section{Operações fundamentais de processamento de dados para aquivos}
   
   Dados são armazenados em memória secundária na forma de arquivos, que são basicamente sequencia de bytes gravados na memória
   secundária. No entanto a aplicação e o hardware tem visões distintas do arquivo.
   
   \subsection{Arquivos lógicos e Arquivos físicos.}
   
   Quando fala-se de um arquivo armazenado em um disco, por exemplo, nos referimos a um coleção de bytes gravados nesse disco. Diz-se então 
   que um arquivo existe fisicamente, e nos referimos a esse conjunto de bytes como um arquivo físico.
   
   No entanto aplicações tem uma visão bastante distinta do que é um arquivo. Um aplicação acessa arquivos por meio do sistema operacional
   e pode apenas ler e escrever dados para o arquivo, mas a aplicação não sabe nada sobre onde está o arquivo. Para a aplicação
   o arquivo é apenas um canal por onde ela pode ler e escrever dados. Desse modo esse canal pode estar conectado a um arquivo físico,
   ou pode estar conectado a um teclado, um socket de rede etc.  Esse canal é utilizado pela aplicação é comumente referenciado 
   como \emph{arquivo lógico}. 
   
   Antes que um programa possa utilizar um arquivo, ele deve solicitar ao sistema operacional para associar um dado arquivo físico com 
   o canal de comunicação utilizado pelo programa. Em várias linguagens de programação isso é feito por meio de um sistema 
   
   \section{Tecnologias de armazenamento}  
   
   Existem, atualmente, diversas alternativas de tecnologias de memória secundária. Cada uma dessas alternativas tem suas próprias peculiaridades,
   se mostrando mais ou menos eficiências para um determinado propósito específico, ao passo em que também podem ser mais ou menos acessíveis financeiramente.
    
   Algumas dessas tecnologias já não são empregadas em larga escala hoje em dia, com exceção de alguns nichos mais específicos, 
   devido a existências de tecnologias melhores. Ainda sim tais tecnologias são apresentadas e discutidas pela sua contribuição histórica.
   
   Outras tecnologias de memória secundária são bem mais recentes, como é o caso das memórias \emph{flash}. Tais dispositivos tem ganhado cada
   vez mais atenção e espaço no mercado devido a seu baixo custo e grande velocidade de operação. Essas tecnologias são discutidas de maneira 
   mais modesta.
   
   Independentemente do tipo de tecnologia empregada para armazenamento e recuperação de informações, é importante ter em mente que o uso de
   estruturas de dados eficientes é de grande importância para que um sistema de processamento de dados apresente um bom desempenho. 
   
   Quando utiliza-se um sistema que tenha que realizar consultas em dados armazenados, espera-se que tal sistema seja capaz de recuperar
   as informações necessárias em um curto intervalo de tempo, também espera-se que tais sistemas sejam capazes de sumarizar as informações
   armazenadas de maneira compacta e compreensível.  Para que se possa atingir esse objetivo é necessária analisar quais operações são
   mais custosas ao processo de recuperação de informação. 
   
   
   A seguir discutimos sucintamente algumas das tecnologias de armazenamento de dados.
   
 
   
   
   \subsection{Fita magnética}
     
     Subtilizadas, mas ainda presentes em algumas empresas e utilizadas como meio de backup. A tecnologia é de armazenamento magnético do dado
     impresso em bandas impregnadas com material ferromagnético. 
     
     Esses dispositivos foram explorados para armazenamento (e comercialização) de áudio, vídeo (fitas VHS) e como meios de backup, devido
     a grande capacidde de armazenamento e baixo custo. 
     
     Não existe um sistema de endereçamento na fita magnética e desse modo  
     
     No entanto o acesso a informação armazenada tem de ser feito de modo sequencial, i.e., para acessar a uma informação armazenada em uma
     dada posição na fita, é necessário passar por todos os dados armazenados fisicamente antes dessa posição. Isso torna fitas magnéticas 
     menos eficientes para utilização em aplicações mais dinâmicas, onde se necessário acessar dados de maneira aleatória.
   
   
   \subsection{Disco Rígido (HD)}
       É dispositivo de armazenamento que utiliza vários discos, empilhados, recobertos com material magnetizável. A leitura e escrita
       é feita por meio de um atuador que se situa próximo a superfície de cada disco. Esse atuador é preso a um braço mecânico móvel de modo que o autador possa 
       ser mover entre a borda externa e o centro do disco. Os discos giram a uma determinada velocidade permitindo que o atuador acesse qualquer região na
       supercície do disco. Ainda é um dos dispositivos de memória secundária principal da maioria dos computadores populares modernos.
   
   \subsection{Disco Ótico}
       
       Tais dispositivos utilizam um laser para imprimir marcas em um substrato abaixo da superfície
       transparente de um disco. Essas marcas podem, posteriormente, serem detectadas pelo laser do dispositivo permitindo assim a leitura dos dados.
       O dispositivo normalmente possui uma plataforma móvel que hospeda um laser. Essa plataforma descreve uma trajetória entre o centro e a borda do disco.
       O disco é encaixado em um motor que o faz girar, permitindo assim que o dispositivo possa acessar qualquer ponto na superfície do disco.
   
   \subsection{Memórias Flash}
       Existem várias tipos de memórias \emph{flash} e esses dispositivos geralmente empregam algum mecanismo elétrico para armazenamento
       dos dados. Tais dispositivos tem se tornado muito atrativos pelo seu baixo tempo de acesso, além disso a ausência de dispositivos 
       mecânicos, como no disco rígido e discos ópticos o torna menos suscetível a falhas mecânicas. Memórias flash ainda são, no entanto, 
       ligeiramente mais caras que as demais tecnologias para serem utilizadas em larga escala. Além disso várias dessas memórias. 
      
  


   
\chapter{Fundamentos de Estruturas de dados para Arquivos}

\begin{lstlisting}[language=java,caption=Calsse java para campo]
  // Calsse campo
  public abstract class Field {
     public abstract void read(ReadStream i); 
     public abstract void write(WriteStream i); 
     public abstract byte[] pack(); 
     public abstract void read(byte b[]); 
  }
\end{lstlisting}



\chapter{Campos e Registros}


\chapter{Ordenação de memória secundária}

\end{document}

